\chapter{Joshua 7}

% \textcolor[cmyk]{0.99998,1,0,0}{
\marginpar{\scriptsize \centering \fcolorbox{blue}{lime}{\textbf{ACHIN FOR A BRUSIN'}}\\ (Joshua 7)

\begin{compactenum}[I.][8]
\index[speaker]{Keith Anthony!Joshua 07 (Achin for a Brusing)}
\index[series]{Joshua (Keith Anthony)!Joshua 07 (Achin for a Brusing)}
\index[date]{2018/03/07!Joshua 07 (Achin for a Brusing) (Keith Anthony)}
\begin{compactenum}[I.]
\item A \textbf{Pernicious Defiling}
\item A \textbf{Painful Defeat} \index[scripture]{Joshua!Jsh 07:04}  (Joshua 7:4) 
\item A \textbf{Public Declaration} \index[scripture]{Joshua!Jsh 07:20}  (Joshua 7:20) 
\item \textbf{Polluted Desire} \index[scripture]{Joshua!Jsh 07:21}  (Joshua 7:21) 
\item A \textbf{Personal Decision} \index[scripture]{Joshua!Jsh 07:21}  (Joshua 7:21) 
\item \textbf{Disproportionate Destruction} \index[scripture]{Joshua!Jsh 07:24}  (Joshua 7:24) 
\item A \textbf{Purifying Death} \index[scripture]{Joshua!Jsh 07:25}  (Joshua 7:25) 
\end{compactenum}}




\footnote{\textcolor[rgb]{0.00,0.25,0.00}{\hyperlink{TOC}{Return to end of Table of Contents.}}}\footnote{\href{https://audiobible.com/bible/joshua_7.html}{\textcolor[cmyk]{0.99998,1,0,0}{Joshua 7Audio}}}\textcolor[rgb]{0.00,0.00,1.00}{Then Jerubbaal, who \emph{is} Gideon, and all the people that \emph{were} with him, rose up early, and pitched beside the well of Harod: so that the host of the Midianites were on the north side of them, by the hill of Moreh, in the valley.}%\footnote{It is interesting that the word ``accursed'' in used thirteen times in Joshua: 1x in Joshua 6:17, 3x in Joshua 6:18, 2x in Joshua 7:1, 1x in Joshua 7:11, 2x in Joshua 7:12, 2x in Joshua 7:13, 1x in Joshua 7:17, and 1x in Joshua 22:20.}\footnote{This is a great chapter on revival, and when it is preached, the title of the sermon is nearly always ``Sin in the Camp.'' It is a picture of how a church, school, or ministry can suffer because of one disobedient Christian. The principle is repeated over and over again in the Scripture. Here the verse says, ``the children of Israel''--that's corporately, as a group --``committed a trespass.'' Well, technically, no. It was just one man: ``for Achan . . . took of the accursed thing.'' One man's sin was charged to the whole group. That's what happened to David in 2 Samuel 24 and 1 Chronicles 21. In his pride, he numbered the men in Israel who were eligible to go to war so he could brag about the size of his army. As a result, thousands of his subjects died due to God's judgment. David prayed, ``I it is that have sinned and done evil indeed; but as for these sheep, what have they done?'' (1 Chronicles 21:17). Well, nothing, but David's sin involved the people. You run into the same thing with Adam. Adam sinned and plunged the whole human race into sin. ``Wherefore, as by one man sin entered into the world, and death by sin; and so death passed upon all men, for that all have sinned'' (Rom. 5:12). That is what is known in systematic theology as ``federal headship.'' A group is judged by the actions and decisions of its leaders. The chief priests rejected Christ and asked for a demon-possessed Roman dictator to be their King (John 19:15); as a result, those Jews have been under the thumb of Rome in one form or another for 2,000 years now. The people were judged by the decisions of their leaders. Now just in case you think that's ``not fair,'' just remember, the way you get into Heaven is on the righteousness of another.}
[2] \textcolor[rgb]{0.00,0.00,1.00}{And the \fcolorbox{black}{bone}{LORD} said unto Gideon, The people that \emph{are} with thee \emph{are} too many for me to give the Midianites into their hands, lest Israel vaunt themselves against me, saying, Mine own hand hath saved me.
[3] \textcolor[rgb]{0.00,0.00,1.00}{Now therefore go to, proclaim in the ears of the people, saying, Whosoever \emph{is} fearful and afraid, let him return and depart early from mount Gilead. And there returned of the people twenty and two thousand; and there remained ten thousand.}%\footnote{Notice the pride and arrogance displayed in verse 3: “Let not all the people go up, but let about two or three thousand men go up and smite Ai; and make not all the people to labour thither; for they are but few.” They are so proud and puffed up that they are telling the commander-in-chief how to conduct the assault. Joshua is the supreme commander, not them. Their job was just to report on Ai, not to advise Joshua on what he ought to do. They were overconfident. The Bible says, “Pride goeth before destruction, and an haughty spirit before a fall” (Prov. 16:18). They thought they could take Ai with a relatively small force; they were about to lose their shirts. The reason they had won at Jericho was not because of their own military prowess; it was because the battle had been ordered by the Captain of the Lord’s host (5:14–6:5). God spoke to Joshua and told him what to do. There is no prayer here, no seeking God’s face in the matter. Instead, Joshua follows the plans of men.}
[4] \textcolor[rgb]{0.00,0.00,1.00}{And the \fcolorbox{black}{bone}{LORD} said unto Gideon, The people \emph{are} yet \emph{too} many; bring them down unto the water, and I will try them for thee there: and it shall be, \emph{that} of whom I say unto thee, This shall go with thee, the same shall go with thee; and of whomsoever I say unto thee, This shall not go with thee, the same shall not go.}%\footnote{Three thousand men of Israel made a direct assault on the gate of Ai, expecting an easy victory, and instead were routed by the men of Ai (later on in 8:25, you will learn that the entire population of Ai, both men and women, was only 12,000, which means that the army of Ai was probably less than 6,000). Thirty-six men died as a result of Achan’s sin. What Achan did affected innocent people. “Shebarim” is more than likely the name of the valley that lay before Ai (see 9:11), as is indicated by the words “in the going down” (vs. 5). The word means “breaks” or “ruins.”}
[5] \textcolor[rgb]{0.00,0.00,1.00}{So he brought down the people unto the water: and the \fcolorbox{black}{bone}{LORD} said unto Gideon, Every one that lappeth of the water with his tongue, as a dog lappeth, him shalt thou set by himself; likewise every one that boweth down upon his knees to drink.}\\
\\
\P \textcolor[rgb]{0.00,0.00,1.00}{And the number of them that lapped, \emph{putting} their hand to their mouth, were three hundred men: but all the rest of the people bowed down upon their knees to drink water.}
[7] \textcolor[rgb]{0.00,0.00,1.00}{And the \fcolorbox{black}{bone}{LORD} said unto Gideon, By the three hundred men that lapped will I save you, and deliver the Midianites into thine hand: and let all the \emph{other} people go every man unto his place.}%\footnote{“Wherefore the hearts of the people melted, and became as water” (vs. 5): the Israelites are ready t o quit. That is how we are. We presume on the grace of God. We get proud and puffed up; then  when the Lord smacks us down, we are ready to quit. Joshua was no exception. When he gets the bad news, he tears his clothes, puts dirt on his head, and falls face down on the ground before the Ark. All those things are Oriental gestures of death. The idea is that it would have been better to die than to get this awful news. Now this was the guy who was ready to take on the giants back there in Numbers 13--14. This is the fellow who challenged the Angel of the Lord back in chapter 5. Look at him now! He’s flat on his face in the dirt, moaning and complaining, “Oh God, why did you ever bring us over Jordan if all you were going to do was let the Amorites kill us? We should just have stayed there in the plains of Moab with the Reubenites, Gadites, and Manassites. What can I say to the people when they are in full retreat before their enemies? All the Canaanites are going to hear about this and surround us and slaughter us, and when they do, Lord, your name is mud.”  Listen to that. He sounds just like those griping, murmuring Jews in Exodus 17:3 and Numbers 11:20, 14:2--3. That’s the low point in Joshua’s life. He has lost his faith. Right here in verses 7--9, Joshua is just as backslidden as the people God killed in the wilderness. Now that’s how the Bible deals with its heroes.” They are not perfect at all. Joshua here sounds just like Elijah does in 1 Kings 19:3–10 (which see). Lester Roloff used to describe it as “throwing a pity party in juniper jungle.” James describes it as being men of “like passions as we are” (James 5:17).}
[8] \textcolor[rgb]{0.00,0.00,1.00}{So the people took victuals in their hand, and their trumpets: and he sent all \emph{the} \emph{rest} \emph{of} Israel every man unto his tent, and retained those three hundred men: and the host of Midian was beneath him in the valley.}
[9] \textcolor[rgb]{0.00,0.00,1.00}{And it came to pass the same night, that the \fcolorbox{black}{bone}{LORD} said unto him, Arise, get thee down unto the host; for I have delivered it into thine hand.}\\
\\
\P \textcolor[rgb]{0.00,0.00,1.00}{But if thou fear to go down, go thou with Phurah thy servant down to the host:}\index[AWIP]{But!Joshua!Jsh 07:010}\index[AWIP]{if!Joshua!Jsh 07:010}\index[AWIP]{thou!Joshua!Jsh 07:010}\index[AWIP]{fear!Joshua!Jsh 07:010}\index[AWIP]{to!Joshua!Jsh 07:010}\index[AWIP]{go!Joshua!Jsh 07:010}\index[AWIP]{down!Joshua!Jsh 07:010}\index[AWIP]{go!Joshua!Jsh 07:010 (2)}\index[AWIP]{thou!Joshua!Jsh 07:010 (2)}\index[AWIP]{with!Joshua!Jsh 07:010}\index[AWIP]{Phurah!Joshua!Jsh 07:010}\index[AWIP]{thy!Joshua!Jsh 07:010}\index[AWIP]{servant!Joshua!Jsh 07:010}\index[AWIP]{down!Joshua!Jsh 07:010 (2)}\index[AWIP]{to!Joshua!Jsh 07:010 (2)}\index[AWIP]{the!Joshua!Jsh 07:010}\index[AWIP]{host!Joshua!Jsh 07:010}\index[NWIV]{17!Joshua!Jsh 07:010}\index[PNIP]{Phurah!Joshua!Jsh 07:010}\footnote{Don’t you know Joshua had a good excuse for that one. “Well, Lord, you said, ‘Humble yourselves therefore under the mighty hand of God,’ so I’m humbling myself. I’m down. You told me to call on your name, so I’m calling. You said to ask, seek, and knock. What do you think I’m doing?”}
[11] \textcolor[rgb]{0.00,0.00,1.00}{And thou shalt hear what they say; and afterward shall thine hands be strengthened to go down unto the host. Then went he down with Phurah his servant unto the outside of the armed men that \emph{were} in the host.}\index[AWIP]{And!Joshua!Jsh 07:011}\index[AWIP]{thou!Joshua!Jsh 07:011}\index[AWIP]{shalt!Joshua!Jsh 07:011}\index[AWIP]{hear!Joshua!Jsh 07:011}\index[AWIP]{what!Joshua!Jsh 07:011}\index[AWIP]{they!Joshua!Jsh 07:011}\index[AWIP]{say!Joshua!Jsh 07:011}\index[AWIP]{and!Joshua!Jsh 07:011}\index[AWIP]{afterward!Joshua!Jsh 07:011}\index[AWIP]{shall!Joshua!Jsh 07:011}\index[AWIP]{thine!Joshua!Jsh 07:011}\index[AWIP]{hands!Joshua!Jsh 07:011}\index[AWIP]{be!Joshua!Jsh 07:011}\index[AWIP]{strengthened!Joshua!Jsh 07:011}\index[AWIP]{to!Joshua!Jsh 07:011}\index[AWIP]{go!Joshua!Jsh 07:011}\index[AWIP]{down!Joshua!Jsh 07:011}\index[AWIP]{unto!Joshua!Jsh 07:011}\index[AWIP]{the!Joshua!Jsh 07:011}\index[AWIP]{host!Joshua!Jsh 07:011}\index[AWIP]{Then!Joshua!Jsh 07:011}\index[AWIP]{went!Joshua!Jsh 07:011}\index[AWIP]{he!Joshua!Jsh 07:011}\index[AWIP]{down!Joshua!Jsh 07:011 (2)}\index[AWIP]{with!Joshua!Jsh 07:011}\index[AWIP]{Phurah!Joshua!Jsh 07:011}\index[AWIP]{his!Joshua!Jsh 07:011}\index[AWIP]{servant!Joshua!Jsh 07:011}\index[AWIP]{unto!Joshua!Jsh 07:011 (2)}\index[AWIP]{the!Joshua!Jsh 07:011 (2)}\index[AWIP]{outside!Joshua!Jsh 07:011}\index[AWIP]{of!Joshua!Jsh 07:011}\index[AWIP]{the!Joshua!Jsh 07:011 (3)}\index[AWIP]{armed!Joshua!Jsh 07:011}\index[AWIP]{men!Joshua!Jsh 07:011}\index[AWIP]{that!Joshua!Jsh 07:011}\index[AWIP]{\emph{were}!Joshua!Jsh 07:011}\index[AWIP]{in!Joshua!Jsh 07:011}\index[AWIP]{the!Joshua!Jsh 07:011 (4)}\index[AWIP]{host!Joshua!Jsh 07:011 (2)}\index[NWIV]{40!Joshua!Jsh 07:011}\index[PNIP]{Phurah!Joshua!Jsh 07:011}\footnote{What was the problem, though? It was sin:
“Israel hath sinned” (vs. 11). Isaiah wrote,
“But your iniquities have separated between
you and your God, and your sins have hid his
face from you, that he will not hear” (Isa.
59:2).Notice He said, “ISRAEL hath sinned.”
He’s not talking about the individual in Israel
who sinned (Achan); He’s talking about the
nation corporately. Israel as a nation was the
son of God in the Old Testament (Exod. 4:22);
outside of Adam and the angels as created sons
of God (see comments on Luke 3:38 in that
Commentary), there were no individual sons of
God in the Old Testament. So God is judging
the nation as a whole here for the sins of one of
its members.
Now although in the New Testament you
are an individual son of God by faith in Jesus
Christ (John 1:12), you still are a member of a
corporate group: the Body of Christ. What you
do as a member of that Body can affect it. Look
at 1 Corinthians 12:25–27.
“That there should be no schism in the
body; but that the members should have the
same care one for another. And whether one
member suffer, all the members suffer with
it; or one member be honoured, all the
members rejoice with it. Now ye are the body
of Christ, and members in particular.”
So what you do, Christian, can affect the
Body of Christ for good or for ill. You see that
all the time when some preacher falls due to a
moral problem; that blackens the eye of the
Church in general before an unsaved world.
Those lost people out there don’t distinguish
between one church and another, or between
one denomination and another. They lump all
the professing Christians together in one group
and say, “Yeah, that’s ‘Christians’ for ya.”
When one preacher sells out the Book, it
affects the whole Body of Jesus Christ. He
influences some student; that student becomes a
professor in a Christian College and affects
whole classes; those students go out as ministers
and cause their congregations not to believe the
Book. The first thing you know, churches all
across the nation have no authority left.
Apostasy is infectious.
When one man gives way in the ranks, the
whole rank waivers and eventually falls to
pieces. I never played football in College (I was
too little for football and too short for
basketball), but I know enough about the game
to realize that if the linemen don’t hold the line,
there is going to be trouble for the men in the
backfield. So every time some preacher or
professor at a Christian College sells out the
Book and substitutes some other authority for
it, he damages the Body of Christ.
“They have also transgressed my
covenant . . . for they have even taken of the
accursed thing.” In the first ten chapters of
Joshua, there are ten cities the Israelites attack.
Jericho is the first of those ten cities, and God
told them that the gold, silver, brass, and iron of
that first city belonged to Him (6:19).
Everything else was to be destroyed because it
was cursed (see our comments on 6:26). The
Israelites were able to take the spoil from the
other nine cities, but the spoil of that first city
belonged to God. So what you are dealing with
here is the tithe, the tenth (see our comments on
Gen. 14:20 in that Commentary). Those Jews
were commanded to give God “the firstfruits”
(Lev. 23:10), and that included “the
firstfruits” of their conquests when they took
the land of Canaan. Jericho was the firstfruits;
it was the tithe.
Leviticus 27:30 says, “all the tithe of the
land . . . is the LORD’S: it is holy unto the
LORD.” When Achan took part of the spoils,
he took the Lord’s tithe and fell under the curse
of Malachi 3:8–9.
“Will a man rob God? Yet ye have
robbed me. But ye say, Wherein have we
robbed thee? In tithes and offerings. Ye are
cursed with a curse: for ye have robbed me,
even this whole nation.”
Notice the word “stolen” in verse 11.
Achan robbed God (Mal. 3:9). He
“dissembled”; that is, he hid it, he concealed
it. And he put it with his own stuff: he treated it
as his when it belonged to God. But the fact
that he hid it showed that he knew he was doing
wrong.}\footnote{\textbf{Ecclesiastes 9:18} -- Wisdom is better than weapons of war: but one sinner destroyeth much good.}\footnote{\textbf{Ecclesiastes 10:1} -- Dead flies cause the ointment of the apothecary to send forth a stinking savour: so doth a little folly him that is in reputation for wisdom and honour.}\footnote{\textbf{Romans 5:15--19} -- But not as the offence, so also is the free gift. For if through the offence of one many be dead, much more the grace of God, and the gift by grace, which is by one man, Jesus Christ, hath abounded unto many. [16] And not as it was by one that sinned, so is the gift: for the judgment was by one to condemnation, but the free gift is of many offences unto justification. 
[17]  For if by one man's offence death reigned by one; much more they which receive abundance of grace and of the gift of righteousness shall reign in life by one, Jesus Christ.) [18] Therefore as by the offence of one judgment came upon all men to condemnation; even so by the righteousness of one the free gift came upon all men unto justification of life. [19]  For as by one man's disobedience many were made sinners, so by the obedience of one shall many be made righteous. }\footnote{\textbf{Galatians 5:9} -- A little leaven leaveneth the whole lump.}
[12] \textcolor[rgb]{0.00,0.00,1.00}{And the Midianites and the Amalekites and all the children of the east lay along in the valley like grasshoppers for multitude; and their camels \emph{were} without number, as the sand by the sea side for multitude.}\index[AWIP]{And!Joshua!Jsh 07:012}\index[AWIP]{the!Joshua!Jsh 07:012}\index[AWIP]{Midianites!Joshua!Jsh 07:012}\index[AWIP]{and!Joshua!Jsh 07:012}\index[AWIP]{the!Joshua!Jsh 07:012 (2)}\index[AWIP]{Amalekites!Joshua!Jsh 07:012}\index[AWIP]{and!Joshua!Jsh 07:012 (2)}\index[AWIP]{all!Joshua!Jsh 07:012}\index[AWIP]{the!Joshua!Jsh 07:012 (3)}\index[AWIP]{children!Joshua!Jsh 07:012}\index[AWIP]{of!Joshua!Jsh 07:012}\index[AWIP]{the!Joshua!Jsh 07:012 (4)}\index[AWIP]{east!Joshua!Jsh 07:012}\index[AWIP]{lay!Joshua!Jsh 07:012}\index[AWIP]{along!Joshua!Jsh 07:012}\index[AWIP]{in!Joshua!Jsh 07:012}\index[AWIP]{the!Joshua!Jsh 07:012 (5)}\index[AWIP]{valley!Joshua!Jsh 07:012}\index[AWIP]{like!Joshua!Jsh 07:012}\index[AWIP]{grasshoppers!Joshua!Jsh 07:012}\index[AWIP]{for!Joshua!Jsh 07:012}\index[AWIP]{multitude!Joshua!Jsh 07:012}\index[AWIP]{and!Joshua!Jsh 07:012 (3)}\index[AWIP]{their!Joshua!Jsh 07:012}\index[AWIP]{camels!Joshua!Jsh 07:012}\index[AWIP]{\emph{were}!Joshua!Jsh 07:012}\index[AWIP]{without!Joshua!Jsh 07:012}\index[AWIP]{number!Joshua!Jsh 07:012}\index[AWIP]{as!Joshua!Jsh 07:012}\index[AWIP]{the!Joshua!Jsh 07:012 (6)}\index[AWIP]{sand!Joshua!Jsh 07:012}\index[AWIP]{by!Joshua!Jsh 07:012}\index[AWIP]{the!Joshua!Jsh 07:012 (7)}\index[AWIP]{sea!Joshua!Jsh 07:012}\index[AWIP]{side!Joshua!Jsh 07:012}\index[AWIP]{for!Joshua!Jsh 07:012 (2)}\index[AWIP]{multitude!Joshua!Jsh 07:012 (2)}\index[NWIV]{37!Joshua!Jsh 07:012}\index[PNIP]{Midianites!Joshua!Jsh 07:012}\index[PNIP]{Amalekites!Joshua!Jsh 07:012}
[13] \textcolor[rgb]{0.00,0.00,1.00}{And when Gideon was come, behold, \emph{there} \emph{was} a man that told a dream unto his fellow, and said, Behold, I dreamed a dream, and, lo, a cake of barley bread tumbled into the host of Midian, and came unto a tent, and smote it that it fell, and overturned it, that the tent lay along.}\index[AWIP]{And!Joshua!Jsh 07:013}\index[AWIP]{when!Joshua!Jsh 07:013}\index[AWIP]{Gideon!Joshua!Jsh 07:013}\index[AWIP]{was!Joshua!Jsh 07:013}\index[AWIP]{come!Joshua!Jsh 07:013}\index[AWIP]{behold!Joshua!Jsh 07:013}\index[AWIP]{\emph{there}!Joshua!Jsh 07:013}\index[AWIP]{\emph{was}!Joshua!Jsh 07:013}\index[AWIP]{a!Joshua!Jsh 07:013}\index[AWIP]{man!Joshua!Jsh 07:013}\index[AWIP]{that!Joshua!Jsh 07:013}\index[AWIP]{told!Joshua!Jsh 07:013}\index[AWIP]{a!Joshua!Jsh 07:013 (2)}\index[AWIP]{dream!Joshua!Jsh 07:013}\index[AWIP]{unto!Joshua!Jsh 07:013}\index[AWIP]{his!Joshua!Jsh 07:013}\index[AWIP]{fellow!Joshua!Jsh 07:013}\index[AWIP]{and!Joshua!Jsh 07:013}\index[AWIP]{said!Joshua!Jsh 07:013}\index[AWIP]{Behold!Joshua!Jsh 07:013}\index[AWIP]{I!Joshua!Jsh 07:013}\index[AWIP]{dreamed!Joshua!Jsh 07:013}\index[AWIP]{a!Joshua!Jsh 07:013 (3)}\index[AWIP]{dream!Joshua!Jsh 07:013 (2)}\index[AWIP]{and!Joshua!Jsh 07:013 (2)}\index[AWIP]{lo!Joshua!Jsh 07:013}\index[AWIP]{a!Joshua!Jsh 07:013 (4)}\index[AWIP]{cake!Joshua!Jsh 07:013}\index[AWIP]{of!Joshua!Jsh 07:013}\index[AWIP]{barley!Joshua!Jsh 07:013}\index[AWIP]{bread!Joshua!Jsh 07:013}\index[AWIP]{tumbled!Joshua!Jsh 07:013}\index[AWIP]{into!Joshua!Jsh 07:013}\index[AWIP]{the!Joshua!Jsh 07:013}\index[AWIP]{host!Joshua!Jsh 07:013}\index[AWIP]{of!Joshua!Jsh 07:013 (2)}\index[AWIP]{Midian!Joshua!Jsh 07:013}\index[AWIP]{and!Joshua!Jsh 07:013 (3)}\index[AWIP]{came!Joshua!Jsh 07:013}\index[AWIP]{unto!Joshua!Jsh 07:013 (2)}\index[AWIP]{a!Joshua!Jsh 07:013 (5)}\index[AWIP]{tent!Joshua!Jsh 07:013}\index[AWIP]{and!Joshua!Jsh 07:013 (4)}\index[AWIP]{smote!Joshua!Jsh 07:013}\index[AWIP]{it!Joshua!Jsh 07:013}\index[AWIP]{that!Joshua!Jsh 07:013 (2)}\index[AWIP]{it!Joshua!Jsh 07:013 (2)}\index[AWIP]{fell!Joshua!Jsh 07:013}\index[AWIP]{and!Joshua!Jsh 07:013 (5)}\index[AWIP]{overturned!Joshua!Jsh 07:013}\index[AWIP]{it!Joshua!Jsh 07:013 (3)}\index[AWIP]{that!Joshua!Jsh 07:013 (3)}\index[AWIP]{the!Joshua!Jsh 07:013 (2)}\index[AWIP]{tent!Joshua!Jsh 07:013 (2)}\index[AWIP]{lay!Joshua!Jsh 07:013}\index[AWIP]{along!Joshua!Jsh 07:013}\index[NWIV]{56!Joshua!Jsh 07:013}\index[PNIP]{I!Joshua!Jsh 07:013}\index[PNIP]{Midian!Joshua!Jsh 07:013}\index[PNIP]{Gideon!Joshua!Jsh 07:013}
[14] \textcolor[rgb]{0.00,0.00,1.00}{And his fellow answered and said, This \emph{is} nothing else save the sword of Gideon the son of Joash, a man of Israel: \emph{for} into his hand hath God delivered Midian, and all the host.}\index[AWIP]{And!Joshua!Jsh 07:014}\index[AWIP]{his!Joshua!Jsh 07:014}\index[AWIP]{fellow!Joshua!Jsh 07:014}\index[AWIP]{answered!Joshua!Jsh 07:014}\index[AWIP]{and!Joshua!Jsh 07:014}\index[AWIP]{said!Joshua!Jsh 07:014}\index[AWIP]{This!Joshua!Jsh 07:014}\index[AWIP]{\emph{is}!Joshua!Jsh 07:014}\index[AWIP]{nothing!Joshua!Jsh 07:014}\index[AWIP]{else!Joshua!Jsh 07:014}\index[AWIP]{save!Joshua!Jsh 07:014}\index[AWIP]{the!Joshua!Jsh 07:014}\index[AWIP]{sword!Joshua!Jsh 07:014}\index[AWIP]{of!Joshua!Jsh 07:014}\index[AWIP]{Gideon!Joshua!Jsh 07:014}\index[AWIP]{the!Joshua!Jsh 07:014 (2)}\index[AWIP]{son!Joshua!Jsh 07:014}\index[AWIP]{of!Joshua!Jsh 07:014 (2)}\index[AWIP]{Joash!Joshua!Jsh 07:014}\index[AWIP]{a!Joshua!Jsh 07:014}\index[AWIP]{man!Joshua!Jsh 07:014}\index[AWIP]{of!Joshua!Jsh 07:014 (3)}\index[AWIP]{Israel!Joshua!Jsh 07:014}\index[AWIP]{\emph{for}!Joshua!Jsh 07:014}\index[AWIP]{into!Joshua!Jsh 07:014}\index[AWIP]{his!Joshua!Jsh 07:014 (2)}\index[AWIP]{hand!Joshua!Jsh 07:014}\index[AWIP]{hath!Joshua!Jsh 07:014}\index[AWIP]{God!Joshua!Jsh 07:014}\index[AWIP]{delivered!Joshua!Jsh 07:014}\index[AWIP]{Midian!Joshua!Jsh 07:014}\index[AWIP]{and!Joshua!Jsh 07:014 (2)}\index[AWIP]{all!Joshua!Jsh 07:014}\index[AWIP]{the!Joshua!Jsh 07:014 (3)}\index[AWIP]{host!Joshua!Jsh 07:014}\index[NWIV]{35!Joshua!Jsh 07:014}\index[PNIP]{God!Joshua!Jsh 07:014}\index[PNIP]{Israel!Joshua!Jsh 07:014}\index[PNIP]{Joash!Joshua!Jsh 07:014}\index[PNIP]{Midian!Joshua!Jsh 07:014}\index[PNIP]{Gideon!Joshua!Jsh 07:014}
[15] \textcolor[rgb]{0.00,0.00,1.00}{And it was \emph{so}, when Gideon heard the telling of the dream, and the interpretation thereof, that he worshipped, and returned into the host of Israel, and said, Arise; for the \fcolorbox{black}{bone}{LORD} hath delivered into your hand the host of Midian.}\index[AWIP]{And!Joshua!Jsh 07:015}\index[AWIP]{it!Joshua!Jsh 07:015}\index[AWIP]{was!Joshua!Jsh 07:015}\index[AWIP]{\emph{so}!Joshua!Jsh 07:015}\index[AWIP]{when!Joshua!Jsh 07:015}\index[AWIP]{Gideon!Joshua!Jsh 07:015}\index[AWIP]{heard!Joshua!Jsh 07:015}\index[AWIP]{the!Joshua!Jsh 07:015}\index[AWIP]{telling!Joshua!Jsh 07:015}\index[AWIP]{of!Joshua!Jsh 07:015}\index[AWIP]{the!Joshua!Jsh 07:015 (2)}\index[AWIP]{dream!Joshua!Jsh 07:015}\index[AWIP]{and!Joshua!Jsh 07:015}\index[AWIP]{the!Joshua!Jsh 07:015 (3)}\index[AWIP]{interpretation!Joshua!Jsh 07:015}\index[AWIP]{thereof!Joshua!Jsh 07:015}\index[AWIP]{that!Joshua!Jsh 07:015}\index[AWIP]{he!Joshua!Jsh 07:015}\index[AWIP]{worshipped!Joshua!Jsh 07:015}\index[AWIP]{and!Joshua!Jsh 07:015 (2)}\index[AWIP]{returned!Joshua!Jsh 07:015}\index[AWIP]{into!Joshua!Jsh 07:015}\index[AWIP]{the!Joshua!Jsh 07:015 (4)}\index[AWIP]{host!Joshua!Jsh 07:015}\index[AWIP]{of!Joshua!Jsh 07:015 (2)}\index[AWIP]{Israel!Joshua!Jsh 07:015}\index[AWIP]{and!Joshua!Jsh 07:015 (3)}\index[AWIP]{said!Joshua!Jsh 07:015}\index[AWIP]{Arise!Joshua!Jsh 07:015}\index[AWIP]{for!Joshua!Jsh 07:015}\index[AWIP]{the!Joshua!Jsh 07:015 (5)}\index[AWIP]{LORD!Joshua!Jsh 07:015}\index[AWIP]{hath!Joshua!Jsh 07:015}\index[AWIP]{delivered!Joshua!Jsh 07:015}\index[AWIP]{into!Joshua!Jsh 07:015 (2)}\index[AWIP]{your!Joshua!Jsh 07:015}\index[AWIP]{hand!Joshua!Jsh 07:015}\index[AWIP]{the!Joshua!Jsh 07:015 (6)}\index[AWIP]{host!Joshua!Jsh 07:015 (2)}\index[AWIP]{of!Joshua!Jsh 07:015 (3)}\index[AWIP]{Midian!Joshua!Jsh 07:015}\index[NWIV]{41!Joshua!Jsh 07:015}\index[PNIP]{Israel!Joshua!Jsh 07:015}\index[PNIP]{LORD!Joshua!Jsh 07:015}\index[PNIP]{Midian!Joshua!Jsh 07:015}\index[PNIP]{Gideon!Joshua!Jsh 07:015}\\
\\
\P  \textcolor[rgb]{0.00,0.00,1.00}{And he divided the three hundred men \emph{into} three companies, and he put a trumpet in every man's hand, with empty pitchers, and lamps within the pitchers.}
[17] \textcolor[rgb]{0.00,0.00,1.00}{And he said unto them, Look on me, and do likewise: and, behold, when I come to the outside of the camp, it shall be \emph{that}, as I do, so shall ye do.}
[18] \textcolor[rgb]{0.00,0.00,1.00}{When I blow with a trumpet, I and all that \emph{are} with me, then blow ye the trumpets also on every side of all the camp, and say, \emph{The} \emph{sword} of the LORD, and of Gideon.}
[19] \textcolor[rgb]{0.00,0.00,1.00}{So Gideon, and the hundred men that \emph{were} with him, came unto the outside of the camp in the beginning of the middle watch; and they had but newly set the watch: and they blew the trumpets, and brake the pitchers that \emph{were} in their hands.}
[20] \textcolor[rgb]{0.00,0.00,1.00}{And the three companies blew the trumpets, and brake the pitchers, and held the lamps in their left hands, and the trumpets in their right hands to blow \emph{withal}: and they cried, The sword of the LORD, and of Gideon.}\index[AWIP]{And!Joshua!Jsh 07:020}\index[AWIP]{the!Joshua!Jsh 07:020}\index[AWIP]{three!Joshua!Jsh 07:020}\index[AWIP]{companies!Joshua!Jsh 07:020}\index[AWIP]{blew!Joshua!Jsh 07:020}\index[AWIP]{the!Joshua!Jsh 07:020 (2)}\index[AWIP]{trumpets!Joshua!Jsh 07:020}\index[AWIP]{and!Joshua!Jsh 07:020}\index[AWIP]{brake!Joshua!Jsh 07:020}\index[AWIP]{the!Joshua!Jsh 07:020 (3)}\index[AWIP]{pitchers!Joshua!Jsh 07:020}\index[AWIP]{and!Joshua!Jsh 07:020 (2)}\index[AWIP]{held!Joshua!Jsh 07:020}\index[AWIP]{the!Joshua!Jsh 07:020 (4)}\index[AWIP]{lamps!Joshua!Jsh 07:020}\index[AWIP]{in!Joshua!Jsh 07:020}\index[AWIP]{their!Joshua!Jsh 07:020}\index[AWIP]{left!Joshua!Jsh 07:020}\index[AWIP]{hands!Joshua!Jsh 07:020}\index[AWIP]{and!Joshua!Jsh 07:020 (3)}\index[AWIP]{the!Joshua!Jsh 07:020 (5)}\index[AWIP]{trumpets!Joshua!Jsh 07:020 (2)}\index[AWIP]{in!Joshua!Jsh 07:020 (2)}\index[AWIP]{their!Joshua!Jsh 07:020 (2)}\index[AWIP]{right!Joshua!Jsh 07:020}\index[AWIP]{hands!Joshua!Jsh 07:020 (2)}\index[AWIP]{to!Joshua!Jsh 07:020}\index[AWIP]{blow!Joshua!Jsh 07:020}\index[AWIP]{\emph{withal}!Joshua!Jsh 07:020}\index[AWIP]{and!Joshua!Jsh 07:020 (4)}\index[AWIP]{they!Joshua!Jsh 07:020}\index[AWIP]{cried!Joshua!Jsh 07:020}\index[AWIP]{The!Joshua!Jsh 07:020}\index[AWIP]{sword!Joshua!Jsh 07:020}\index[AWIP]{of!Joshua!Jsh 07:020}\index[AWIP]{the!Joshua!Jsh 07:020 (6)}\index[AWIP]{LORD!Joshua!Jsh 07:020}\index[AWIP]{and!Joshua!Jsh 07:020 (5)}\index[AWIP]{of!Joshua!Jsh 07:020 (2)}\index[AWIP]{Gideon!Joshua!Jsh 07:020}\index[NWIV]{40!Joshua!Jsh 07:020}\index[PNIP]{LORD!Joshua!Jsh 07:020}\index[PNIP]{Gideon!Joshua!Jsh 07:020}\footnote{Two things about Achan’s confession here: First, notice that Achan acknowledges that he sinned against God. Do you know why that part of the confession is so important? Because Mohammed and “Allah” in the Koran say it’s a lie. Sura 4:111 says, “And whoever commits a sin, he only commits it AGAINST HIS OWN SOUL.” Whoever “the god” was (that’s what the word Allah means—“the god”), he certainly was not the God who wrote the Bible. In the Holy Bible (not the “Noble Koran”), sin is against God. Joseph is worried about sinning against God (Gen. 39:9). David confessed, “I have sinned against the LORD” (2 Sam. 12:13). The prodigal said, “I have sinned against heaven” (Luke 15:21). Even backslidden Jews like the ones in 1 Samuel 7:6 and Hell-bound pagans like Pharaoh in Exodus 10:16 had enough sense to realize they sinned against God. Only someone as stupid as Allah, “Gabriel” (the 600-winged demon who dictated the Koran to Mohammed, not the angel who appeared to Zacharias and Mary in Luke 1), and Mohammed wouldn’t know that sin is against God. You might do yourself or someone else wrong, and the Bible certainly does use the phrase “sinned against” in that respect from a practical point of view (see Judg. 11:27; 1 Sam. 19:4, 24:11). But theologically and doctrinally speaking, for sin to be “sin,” it has to be against God, because “sin is the transgression of the law” (1 John 3:4) —God’s Law. The second thing about Achan’s confession in verse 20 is that you can’t always tell what a man means when he says, “I have sinned.” He might not mean “I want to get right with the Lord” at all. Here, Achan said, “I have sinned” because he was caught red-handed. What else could he do? When Pharaoh said “I have sinned” in Exodus 9:27, 10:16, what he meant was: “I am sick and tired of these plagues; now get off my back.” When Saul said “I have sinned” (1 Sam. 15:30), he meant: “Let me go on being King as if nothing ever happened.” When Balaam said “I have sinned” (Num. 22:34), what he actually meant was: “Please don’t kill me.” I mean, the Angel of the Lord was standing right there in the way with a drawn sword ready to take off his head (Num. 22:31). When Judas said “I have sinned” (Matt. 27:4), all he meant was: “I betrayed an innocent man, and my conscience is killing me over it.” Of course, not all confessions of sin in the Bible are like that. When the prodigal said, “I have sinned” (Luke 15:18–21), he intended to go home where he belonged and make things right with daddy. When David said “Against thee, thee only, have I sinned, and done this evil in thy sight” (Psa. 51:4), he was down on his face getting right with God. Brethren, the Lord knows what a man means when he says “I have sinned.” I have the testimony of a Limey in World War I who got caught in the artillery zone between his own side and the Germans. He was lying there on his face in the mud, seeing pieces of men blown every place all around him. He said he confessed to God everything he could think of, and he said that all the time he was doing that, he had no intention of stopping any of it. He was just trying to cut a deal with God to save his worthless hide. His saying “I have sinned” didn’t carry any weight. The Lord knows when the confession is real and when it isn’t.}
[21] \textcolor[rgb]{0.00,0.00,1.00}{And they stood every man in his place round about the camp: and all the host ran, and cried, and fled.}\index[AWIP]{And!Joshua!Jsh 07:021}\index[AWIP]{they!Joshua!Jsh 07:021}\index[AWIP]{stood!Joshua!Jsh 07:021}\index[AWIP]{every!Joshua!Jsh 07:021}\index[AWIP]{man!Joshua!Jsh 07:021}\index[AWIP]{in!Joshua!Jsh 07:021}\index[AWIP]{his!Joshua!Jsh 07:021}\index[AWIP]{place!Joshua!Jsh 07:021}\index[AWIP]{round!Joshua!Jsh 07:021}\index[AWIP]{about!Joshua!Jsh 07:021}\index[AWIP]{the!Joshua!Jsh 07:021}\index[AWIP]{camp!Joshua!Jsh 07:021}\index[AWIP]{and!Joshua!Jsh 07:021}\index[AWIP]{all!Joshua!Jsh 07:021}\index[AWIP]{the!Joshua!Jsh 07:021 (2)}\index[AWIP]{host!Joshua!Jsh 07:021}\index[AWIP]{ran!Joshua!Jsh 07:021}\index[AWIP]{and!Joshua!Jsh 07:021 (2)}\index[AWIP]{cried!Joshua!Jsh 07:021}\index[AWIP]{and!Joshua!Jsh 07:021 (3)}\index[AWIP]{fled!Joshua!Jsh 07:021}\index[NWIV]{21!Joshua!Jsh 07:021}\footnote{Verse 21 is one of the passages in the Scriptures that gives you the progression of sin. Another one is James 1:15. The progression of sin is first presentation; here, “when I SAW among the spoils.” The next step is illumination: the Holy Spirit gives light on the temptation on whether it is right or wrong. Achan had that illumination back in 6:18--19. So far, so good. Sin does not come with the presentation or the illumination. Where sin enters is when you debate the matter in your mind: should I do it or should I not? Debate shows a degree of unbelief in the light God has given you on the matter. Debate is trying to figure out whether you can commit the sin with little or no consequences. Debate is where sin enters. I guarantee you that when Achan saw that beautiful Babylonian garment, that sack of silver, and that wedge of gold that he thought to himself, ``Man, it’s a shame I have to burn that nice coat. And look at all that money. All they ever talk about down at the Tabernacle is money, money, money. I can think of a few things I could spend that money on instead of those priests down at the Tabernacle.'' That’s the kind of stuff that went through Achan’s head. Sin enters with debate. Achan looked; then he coveted. The term is “lust” in James 1:15--“Then when lust hath conceived, it bringeth forth sin.” The sin Achan’s lust brought forth was theft (“then I coveted them, and took them”). And just like James 1:15 says, Achan’s lust and sin lead to his death (vs. 25). There is an interesting cross-reference to the “wedge of gold” in verse 21. It is found in Isaiah 13:12. “I will make a man more precious than fine gold; even a MAN than the GOLDEN WEDGE of Ophir.” There, certain men are compared to gold. The first time gold shows up in your Bible, it is connected to the place where man was created (Gen. 2:7–12). When you get to the end of Revelation, you find a city (the New Jerusalem) of “pure gold” (Rev. 21:18, 21). That city is called “the bride, the Lamb’s wife” (Revelation 21:9–10). Why, in Ephesians 5:25–32, “the Lamb’s wife” is the Church, the Body of Jesus Christ made up of saved believers in this age. But if I were to tell you that in the resurrection in Eternity, your glorified body would be made of living gold, some of the brethren would “blow a gasket” over “Ruckman’s peculiar doctrines,” so we will leave the discussion right there.}\\\
\\
\P \textcolor[rgb]{0.00,0.00,1.00}{And the three hundred blew the trumpets, and the \fcolorbox{black}{bone}{LORD} set every man's sword against his fellow, even throughout all the host: and the host fled to Beth-shittah in Zererath, \emph{and} to the border of Abel-meholah, unto Tabbath.}
[23] \textcolor[rgb]{0.00,0.00,1.00}{And the men of Israel gathered themselves together out of Naphtali, and out of Asher, and out of all Manasseh, and pursued after the Midianites.}
[24] \textcolor[rgb]{0.00,0.00,1.00}{And Gideon sent messengers throughout all mount Ephraim, saying, Come down against the Midianites, and take before them the waters unto Beth-barah and Jordan. Then all the men of Ephraim gathered themselves together, and took the waters unto Beth-barah and Jordan.}%\footnote{Verse 24 shows you that when Israel crossed over Jordan, more than just the 600,000 armed men crossed over. The families of the nine-and-a-half tribes that would receive their inheritance in Canaan came over as well. So nearly a million-and-a-half people cross over Jordan on dry ground. A landslide (see our remarks under 3:13)?! Go kid your grandmother; we have better sense. Now there are a couple of important lessons here. First, “the love of money is the root of all evil” (1 Tim. 6:10). That has been
changed in all the new “Bibles” to “the love of money is a root of all kinds of evil.” The
excuse is that the definite article is not there in
Greek. Of course, that is a “red herring” to get
you to overlook the fact that all translations add
definite articles that aren’t in the original Greek
and Hebrew. You take 1 Corinthians 2:16
—“But we have the mind of Christ.” Why,
there’s no definite article in front of the Greek
word for “mind” (noun), yet every translation
adds one. Why? Because it would be nonsense
to say “we have A mind of Christ.” So don’t
accept the word of these hypocrites who will
criticize the King James for doing something
and then turn right around and do the same
thing themselves.In this age, “the love of money is the root
of all evil.” That “love of money” starts out in
the Old Testament as the sin of covetousness,
and that sin certainly is “the root of all evil.”
The Devil started out coveting the throne of
God, and that covetousness turned into pride,
thinking he could actually do it. Covetousness is
the basic root sin, and “the love of money”
springs from covetousness. Paul says,
“covetousness . . . is idolatry” (Col. 3:5);
that’s because you are putting something you
desire in the place of God and what God said.
That’s what Achan did.
The second important lesson here is that sin
affects innocents. Not only did 36 men die because of Achan’s sin, Achan caused the death of his wife, his children, and his beasts also. Achan’s sin affected people who had nothing to do with it: it affected the congregation (vs. 13), it affected the spiritual leader (vs. 7), and it affected Achan’s family (vs. 24). But the most important lesson that we learn from this passage is that sin committed should be judged and confessed, not hidden away. The New Testament commentary on this point is: “if we would judge ourselves, we should not be judged” (1 Cor. 11:31). If you don’t want the Lord to judge you, you had better take care of the thing yourself.}
[25] \textcolor[rgb]{0.00,0.00,1.00}{And they took two princes of the Midianites, Oreb and Zeeb; and they slew Oreb upon the rock Oreb, and Zeeb they slew at the winepress of Zeeb, and pursued Midian, and brought the heads of Oreb and Zeeb to Gideon on the other side Jordan.}%\footnote{“And Joshua said, Why hast thou troubled us? the \fcolorbox{black}{bone}{LORD} shall trouble thee this day.” The word “Achor” (vs. 24) means “trouble.” Over in 1 Chronicles 2:7, Achan is called “Achar, the troubler of Israel”; “Achar” means “troubler.”
“And all Israel stoned him with stones, and burned them with fire, after they had
stoned them with stones.” Rough chapter. That’s capital punishment for stealing. The Old
Testament prescribed capital punishment for
more than just murder. There was capital
punishment for adultery, for sexual perversion,
for incorrigible disobedience to and disrespect
for one’s parents, for idolatry, and for
involvement in the occult. If the whole Law
were applied to America today, you wouldn’t
be able to travel from your home to your job
without stepping on dead bodies.
Now there are a couple of reasons the
punishment is so severe. The first is to show
you just how holy God is. Americans think God
is akin to Santa Claus or a doting grandfather:
all He wants to do is good to you, and He
would never punish you for your sin, at least
not in Hell. Listen, God is so holy that He
would give you capital punishment for stealing
fruit off a tree (Gen. 3). “The wages of sin is
death” (Rom. 6:23).
Second, God judged Achan’s sin so
severely because it made Israel’s motive for
conquering the land look suspect. The heathen
looking at what Achan did would say that the
Israelites were just marauders coming into the
land to steal and pillage, like the barbarian tribes
did to Rome or like what the Vikings did to
England. No spiritual purpose would be
ascribed to the Israelites conquering the land;
God would have nothing to do with it.
One of the reasons God didn’t let His
people take the spoils from Jericho was to show
the heathen there in Canaan that He didn’t call
the Jews out of Egypt to steal somebody else’s
property. The conquest of the land of Canaan
was divine judgment against the gross sins being
committed by the heathen there (see our
remarks in the Introduction). Jericho illustrated
both the spiritual motive and power behind the
conquest of the land, and the severity of the
judgment against Achan’s sin put the quietus on
the heathen ascribing the motive to
covetousness.
Now that is something to keep in mind in
Christian work. What is the motivation for more
land, more buildings, more TV or radio
stations, etc.? You have to make sure that the
motive is not the almighty dollar, because the
world won’t hesitate to judge you as mercenary
in such matters. Christian work should be done
in such a way that nobody can legitimately
question its motivation.}
[26] \textcolor[rgb]{0.00,0.00,1.00}{And they raised over him a great heap of stones unto this day. So the \fcolorbox{black}{bone}{LORD} turned from the fierceness of his anger. Wherefore the name of that place was called, The valley of Achor, unto this day.}%\footnote{A mound of stones was raised over the charred remains of Achan, his family, his beasts, and his goods. Years later at the end of the book of Joshua, anybody walking by could see that pile of rocks and remember why it was there and be warned not to covet and steal from God. The pile of stones over Achan should be set in contrast with the pile of stones brought up from the River Jordan back in chapter 4 (see our remarks in that chapter). Those stones hauled out of the Jordan and stacked up on the bank signified death to the old life and resurrection to a new life. The stones piled over Achan are a picture that the old life should have been left dead and buried at the bottom of the Jordan, and shouldn’t have been up parading around coveting this and coveting that. The New Testament commentary on this is Romans 8:13. “For if ye live after the flesh, ye shall die: but if ye through the Spirit do mortify the deeds of the body, ye shall live.” Back to Joshua 7:26—“So the \fcolorbox{black}{bone}{LORD} turned from the fierceness of his anger. Wherefore the name of that place was called, The valley of Achor, unto this day.” In fact, it’s known as “The valley of Achor” all the way up until 740 B.C., and although some of those Bible names for places have been lost through the passage of time, the Lord hasn’t forgotten about them. You see, there are two other passages of Scripture that deal with this valley; both are connected with the restoration of Israel at the Second Advent. “Thus saith the LORD, As the new wine is found in the cluster, and one saith, Destroy it not; for a blessing is in it: so will I do for my servants’ sakes, that I may not destroy them all. And I will bring forth a seed out of Jacob, and out of Judah an inheritor of my mountains; and mine elect shall inherit it, and my servants shall dwell there. And Sharon shall be a fold of flocks, and THE VALLEY OF ACHOR a place for the herds to lie down in, for my people that have sought me” (Isa. 65:8–10). “Therefore, behold, I will allure her [talking about the nation of Israel that had been put away by God], and bring her into the wilderness, and speak comfortably unto her. And I will give her her vineyards from thence, and THE VALLEY OF ACHOR for a door of hope: and she shall sing there, as in the days of her youth, and as in the day when she came up out of the land of Egypt. [When does that happen? Look at the next verse] And it shall be at THAT DAY [a reference to the Second Advent--see Isa. 2:11, 17, 4:2, 10:20, 27, 11:10–11], saith the LORD, that thou shalt call me Ishi, and shalt call me no more Baali” (Hosea 2:14--16). So in the place where a terrible judgment was executed to turn the Lord “from the fierceness of his anger,” Israel will receive a blessing when Christ returns at the Second Advent. The Lord is able to take the worst circumstance and make something good out of it. The spiritual application to the born-again child of God is Romans 8:28. “And we know that all things work together for good to them that love God, to them who are the called according to his purpose.”}

