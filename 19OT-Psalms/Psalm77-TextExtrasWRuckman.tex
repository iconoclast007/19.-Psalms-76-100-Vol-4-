\chapter{Psalm 77}

\footnote{\textcolor[rgb]{0.00,0.25,0.00}{\hyperlink{TOC}{Return to end of Table of Contents.}}}
\textcolor[rgb]{0.00,0.00,1.00}{To the chief Musician, to Jeduthun, A Psalm of Asaph.}\\
\\
\textcolor[rgb]{0.00,0.00,1.00}{I cried unto God with my voice, \emph{even} unto God with my voice; and he gave ear unto me.}\footnote{[RUCKMAN] Verse 1 is self explanatory (see Psalm 5:1, 17:1, and comments).\cite{Ruckman1992Psalms}}
[2] \footnote{[RUCKMAN] Verse 2 needs little explanation. The “sore” that “ran in the night” is from the “loathsome disease” mentioned in Psalm 38:7, which see. In application, there are thousands of saints whose “bed sores” and “running sores” kept them in misery weeks at a time. When the “Champions for Christ” at Lynchburg face this common place truth of life, they “head for the hills.” Kroll says, “It appears this clause is better translated....” (Ho hum; so what’s new? Time magazine: “God is dead.” Oral Roberts: “God is not dead.” God: “Who is Oral Roberts?”) Well, who is “Kroll” but one more link coming out of a baloney sausage factory that has been putting out destructive critics for one hundred years. “My hand was stretched out in the night,” says Kroll, aping the Reversed Vision (RV), the Nutty Retarded Vision (RSV), the Living Blockheads with the Nicolaitans Imitation Version (NIV), and the usual: ASV, NASV, NKJV, etc. If you have a running sore, you will find your case documented in one Bible and one Bible only. It is not in any English Bible, but an Authorized Version. “Dead Duck Otherism” fails to mention it. Like 2 Timothy 2:15; 1 Thessalonians 5:23; and 2 Corinthians 2:17, there remains only “King James Onlyism” to give you the whole truth; any other version will LIE to you just as quickly as Bob Jones III, Bob Jones IV, or Bob Jones V. “Dead Duck Otherism.” The reading of the ASV, NASV, NKJV, RSV, and NRSV comes from spiritualizing “the sore.” That is, the “sore” is not a literal sore, even though the Hebrew word is describing something “flowing out.” This is altered to “stretched out” to make the verse figurative. Leaving the text as it stands, we have an “issue of blood” (Lev. 12:7) or pus (Lev. 15:2) that will not dry up. It is a dripping infection coming from a limb; in this case, probably from the hands.\cite{Ruckman1992Psalms} 
\begin{compactenum}
\item “I sought the Lord” (vs. 2).
\item “I remembered God” (vs. 3).
\item “I complained” (vs. 3).
\item “My spirit was overwhelmed” (vs. 3).
\item “I am...troubled” (vs. 4).
\item “I have considered” (vs. 5).
\item “I call to remembrance” (vs. 6).
\item “I commune with mine own heart” (vs. 6).
\item “My spirit made diligent search”(vs. 6).
\end{compactenum} }
[3] \textcolor[rgb]{0.00,0.00,1.00}{I remembered God, and was troubled: I complained, and my spirit was overwhelmed. Selah.}
[4] \textcolor[rgb]{0.00,0.00,1.00}{Thou holdest mine eyes waking: I am so troubled that I cannot speak.}
[5] \textcolor[rgb]{0.00,0.00,1.00}{I have considered the days of old, the years of ancient times.}
[6] \textcolor[rgb]{0.00,0.00,1.00}{I call to remembrance my song in the night: I commune with mine own heart: and my spirit made diligent search.}
[7] \textcolor[rgb]{0.00,0.00,1.00}{Will the Lord cast off for ever? and will he be favourable no more?}\footnote{[RUCKMAN] Six questions are asked, and all six resemble each other. The first is, “Will the Lord cast off for ever?” Yes, He will, in some cases (1 Chron. 28:9). Where He promised not to (vs. 8), as in Jeremiah 33:26 and 2 Samuel 7:15, He will NOT. “Will he be favorable no more?” Notice how we are moving into references that deal with Israel, and not just the “sweet Psalmist.” This is confirmed immediately after “Hath he in anger shut up his tender mercies?” with the familiar watchword we have come to know so well: “Selah.” Observe that in Jeremiah’s time--one of the greatest pictures of the Antichrist’s future destruction of Jerusalem (Rev. 11)--the “mercies” are taken away (Jer. 16:5). We are back at Lamentations 3:22 again, having been here a dozen times since Psalm 1. From the standpoint of Israel, they must remember God’s mercies in the past; note how Ezra, Joshua, and Nehemiah harp on this note (Ezra 9:7--13; Josh. 24:4--11; Neh. 9:7--26). The past is the monitor of the future. Thus we read “I will remember the works of the LORD” and “thy wonders of old.”\cite{Ruckman1992Psalms}}
[8] \textcolor[rgb]{0.00,0.00,1.00}{Is his mercy clean gone for ever? doth \emph{his} promise fail for evermore?}
[9] \textcolor[rgb]{0.00,0.00,1.00}{Hath God forgotten to be gracious? hath he in anger shut up his tender mercies? Selah.}
[10] \textcolor[rgb]{0.00,0.00,1.00}{And I said, This \emph{is} my infirmity: \emph{but} \emph{I} \emph{will} \emph{remember} the years of the right hand of the most High.}\footnote{[RUCKMAN]``And I said, This is my infirmity” (vs. 10). You didn’t say that in any Bible but one. The NKJV does not have the reading neither does the RV, RSV, NRSV, ASV, NASV, NIV, or any other apostate corruption parading around as a “Bible.” An infirmity is a “weakness,” as in 2 Corinthians 12:5, 11:30. The Psalmist has two weaknesses: one, a physical sickness that keeps him awake and makes him complain; and two, a spiritual weakness that makes him doubt God’s mercies and His promises (vss. 8--9). Jeremiah is a perfect illustration. He gets so “put out” with God’s lot for his life and God’s dealings with him (Jer. 20:7), that he is tempted to charge God with lying (Jeremiah 15:18). (Oh yeah, man! You can’t get this stuff in any Christian college!) The Dead Dodo says--that’s the one Jerry Falwell says he reads every morning (TLB) and says that it is responsible for the spiritual success of his ministry!--“this is my fate.” The Communist version of the NCC says “this is my grief.” The NIV says ``to this will I appeal.'' Hupfeld and Perowne want it to read “my woe.” Hitzig, Ewald, and DeWette make it “my cross” (reading the New Testament back into it, with complete disregard for any Hebrew text or any Hebrew copy of any Hebrew manuscript). Literally, the word in Hebrew is “disease”, showing that the AV handling of verse 2 (“my sore”) was the right reading. The idea is simple, so most of the commentators simply read the “idea” into the verse and then alter the verse so it will fit their explanation of it: a typical Alexandrian LXX Origenistic operation. The trouble is not with God; the trouble is with ME. The accent is on “my.” It is my infirmity, which I earned and have deserved, so there is no point in doubting God (vss. 7--9) and claiming He has changed simply because MY condition has changed. “Selah.” He determines to go on “witnessing” (“I will...talk of thy doings.”) even if he is “weak” (see 2 Cor. 12:10, 11:29).\cite{Ruckman1992Psalms}}
[11] \textcolor[rgb]{0.00,0.00,1.00}{I will remember the works of the LORD: surely I will remember thy wonders of old.}\footnote{[RUCKMAN] If one will read this with the previous verse and then compare them both with Luke 12:20 and 1 Timothy 6:9--15, he will find tremendous cohesion and “relevancy” (twentieth century cliché) in the Scripture. The “rich man” of the verse is the rich man of Luke 12 or Luke 18, who supposes that money is a substitute for the “name” mentioned in verse 10 (cf. remarks on 11:4). The “rich man” looks at the lonely tower on the hill, smiles, and says, “When the enemy comes in, he will take that tower; but he’ll never get me, because I own this city and it has a wall around it as high as that one tower.” [When the enemy comes in (1 Kings 22), down comes the wall (Josh. 6:20) and up goes the whole city in flames (Judg. 20:40--48)!] The next verse (vs. 12) fits beautifully. “Before destruction the heart of man is haughty, and before honour is humility” (cf. detailed analysis under 16:18 and 15:33). The proud man who will not humble himself will not run to the tower of verse 10. The “righteous” (rich or poor) will humble himself and “flee from the wrath to come” (Luke 3:7); in the day of judgment he will find that there is one Tower that no demon, devil, angel, principality, power, or fury ever was able to conquer. It was built by a man who “counted the cost” (Luke 14:28) before He began to build (Eph. 2:22), and its foundation (1 Cor. 3:11) is made out of the same material as the capstone (Matt. 21:42): pure gold, mined, refined, shaped, and placed without hands (Dan. 2:45). The Septuagint aborts the text again, and its alteration is not worth repeating or commenting on. The ASV alters “conceit” (vs. 11) to “imagination” for the sake of CLARITY, and then proceeds to use the same word--“conceit”--for the same meaning in a similar passage in the New Testament (Rom. 12:16)! If “imagination” is clearer, why was it rejected by the men who used it to “clarify” the AV (1611)? Answer: no reviser since 1880 was ever interested in clarity or accuracy, and their own pens condemned them and signed their death warrants (see comments under 16:27).\cite{Ruckman1992Psalms}}
[12] \textcolor[rgb]{0.00,0.00,1.00}{I will meditate also of all thy work, and talk of thy doings.}
[13] \textcolor[rgb]{0.00,0.00,1.00}{Thy way, O God, \emph{is} in the sanctuary: who \emph{is} \emph{so} great a God as \emph{our} God?}\footnote{[RUCKMAN] Now Bible doctrines start seeping into the personal problems, for up shows ``our God''--the God of Israel (see Psalm 20:5, 7). Up show “the sons of Jacob and Joseph” (vs. 15) with the Second Advent reference given the second time: “Selah” (vs. 15). The Psalmist’s personal problems are therefore the problems of a Tribulation Israelite. His condition is the same: sleepless and speechless. His doubts are the same (see vss. 6–9). His meditation will have to be the same (see Lam. 3). Israel’s hope in the Tribulation is that God will repeat what He did in Exodus and Joshua, and that is exactly what He will do. This explains the word “Jesus” appearing for “Joshua” in Hebrews 4:8, and it explains why Moses returns in the middle of the Tribuation (Rev. 11). Kroll bombs out completely; all he can say about the God of Israel (“our God”) is “God is great, God is good.” Man what a “nugget”! Why not the rest of it? “Lord we thank thee for this food?” (How about “Roses are red, violets are blue; always have been and always will be”?) To eliminate the God of Israel--the God of Abraham, Isaac, and Jacob--Kroll grabs verse 14 and says that “other nations” in the past have seen the “power of Jehovah,” so He is “awesome” in their sight. Well, not in 500 B.C., 300 B.C., 200 B.C. or in A.D. 300, A.D. 400, A.D. 500, or A.D. 600, 700, 1000, 1200, 1500, 1800, 1900, no. One reference is given, and you know where it will have to be--the Red Sea crossing (see Psalm 68 and comments). Once again, Kroll and all the commentators are going to relegate the Second Advent to the PAST (vss. 19--20) while Asaph is prophesying. No ``waters saw thee, O God'' in Exodus 15, and the “depths” in both places (Psalm 77 and Exodus 15) were references to the ``depths'' of Psalm 68:22 and Habakkuk 3:15. But there is no way to cure the apostates of their dead “historicizing” tendencies. They WILL “limit the Holy One,” and they will stick to the obvious past at the expense of future revelation.\cite{Ruckman1992Psalms}}
[14] \textcolor[rgb]{0.00,0.00,1.00}{Thou \emph{art} the God that doest wonders: thou hast declared thy strength among the people.}\footnote{[RUCKMAN] It is the God of Israel who does “wonders” (vs. 14). The “god” of Germany, the USA, Russia, India, China, Japan, England, Spain, Italy, France, and Greece doesn’t really do much of anything. He cannot put out the sun (Exod. 10:21--23), he cannot dry up oceans (Exod. 14), he cannot dry up rivers (Josh. 3:13--16), he cannot produce lice and flies (Exod. 8:16), he cannot lengthen a day (Josh. 10:13), he cannot prophesy details five hundred years ahead of time, he cannot atone for sins, he cannot resurrect himself, and he cannot cause one nation of eight million people (Israel) to whip the United Nations after they have mustered two hundred million armed cavalrymen (Rev. 16:12--16). There are a number of other things that the “god” of other nations cannot do; these are just a handful. God’s “arm” shows up again (see Isa. 30:30, 40:10) and His “right hand” (vs. 10), which are references to the Lord Jesus Christ in the Old Testament. “The sons of Jacob and Joseph” would include Ephraim and Manasseh as part of the “twelve tribes.” You will notice, if you read the commentators, the constant confusing of Exodus 15 with events that will take place in the next ten years. The trick is to KILL the truth; cover up divine revelation and then pretend that you have “milked” all of the truth from a passage by confining it to something that is already done and already known.\cite{Ruckman1992Psalms}}
[15] \textcolor[rgb]{0.00,0.00,1.00}{Thou hast with \emph{thine} arm redeemed thy people, the sons of Jacob and Joseph. Selah.}
[16] \textcolor[rgb]{0.00,0.00,1.00}{The waters saw thee, O God, the waters saw thee; they were afraid: the depths also were troubled.}\footnote{[RUCKMAN] You can expect a mass defection from the truth. There will be ``conscientious objectors'' and deserters from the ranks right and left, and the Funnymentalists will lead the way. ``The waters saw thee, O God, the waters saw thee...the depths.'' We are again back in Genesis 1:6--9 where all of the commentators, all of the scholars, all of the revisors, and all of the Bible teachers altered the Holy Bible to bring it down to their level of ignorance. This is Psalm 68:22; Job 41:31--32; and Psalm 104:3 all over again, describing a thing that has not taken place yet anywhere on this earth. The Red Sea crossing is only a type, and even in the record of the type, the inspired writer sticks in phrases that no one could apply literally to what was then taking place (see comments on Exodus 15:1--12 in that commentary). \cite{Ruckman1992Psalms}}
[17] \textcolor[rgb]{0.00,0.00,1.00}{The clouds poured out water: the skies sent out a sound: thine arrows also went abroad.}\footnote{[RUCKMAN] The “arrows” of verse 17 are the “arrows” of Habakkuk 3:11 and Psalm 18:14. The “clouds” pouring out water are the clouds of 2 Samuel 22:12; 1 Kings 18:45; and James 5:7. The “sound” is the one mentioned in Psalm 47:5 and 89:15. It is heard in Psalm 50, Habakkuk 3, and Hebrews 12; but Kroll, Baethgen, Yates, Motyer, Davidson, Briggs, Lange, Clarke, Ellicott, Spurgeon, Gerhous, Calvin, Diodati, Caryl, Mant, Rogers, Hooper, Andrews, Dummelow, Webber, Lindsay, Kirban, Rosenthal, Church, and Bullinger are deaf mutes. Bullinger, the hyper-Dispensationalist, assures us that the “waters” are the Nile River one time and the Red Sea the next time, and that a literal belief in a literal “abyss” is nothing but “Babylonian mythology” (Companion Bible, p. 795). “Caution: Adults playing.” There is nothing like thirty years of formal education in Hebrew and Greek at a “recognized” Christian seminary to produce a spiritual guide who is just as blind as a mole.\cite{Ruckman1992Psalms}}
[18] \footnote{[RUCKMAN] The “thunder” of verse 18--not heard once at the Red Sea crossing--is the thunder of Psalm 18:13, 29:3 (which see). The trembling of the earth takes place in Revelation 6:12, 8:5, and Revelation 11:13, 19, in a way that would make the Red Sea crossing look like a minor tremor on the “Richter scale,” and “the sea” of verse 19 is the “sea” of Job 41:31--32 and Revelation 21:1. All of the commentators miss all the references. Want some samples of ``historic positions'' from “Funnymentalists”? “The waters of the Red Sea...are here beautifully represented as endued with sensibility, as seeing, feeling, and being confounded, even to the lowest depth...this is in fact true POETRY...there are no poets who can vie with those of the Hebrew nation.” (``Thirty days hath September, October, April, June, and No Wonder; all the rest eat peanut butter, except Grandma, and she rides a bicycle.'')  ``The depths are mentioned in addition to the waters to show that the dominion and power of God reach not only to the surface of the waters but penetrate to the most profound abysses and agitate and restrain the waters from their lowest bottom.'' (Bumper stickers: ``The world is flat; Class of 1491,'' ``Kilroy wouldn't be caught dead in this dump.'') ``These verses, which act as a hymn within a hymn, differ greatly in mood and form from the rest of the poem...the dominant note of this second section is God's power over nature in general, the position of the passage $\hdots$'' (``Roses are red, violets are blue; if we're both `schizo,' that makes four of us.'')  ``A poetic development of the divine mastery of the forces of nature in overthrowing the Egyptians in the sea...probably a reference to the returning waters which obliterated from sight, but not from memory, the path taken by the cloudy fiery pillar.'' (Notice: ``Please don't write on the wall.'' Written below: “Whatdya want me to do? Type on it?”) “When we successfully traverse the waters of despair we will be able to look back and recognize that we have been led through by the Shepherd and Bishop of our souls.” And what about sound doctrine? And what about the Scriptures being written primarily for “doctrine” (2 Tim. 3:16)? These characters can only digest MILK, and skim milk at that. Bread and meat are not found in their diet. The truth is that the Lord Jesus Christ, at the Second Advent, comes down through the “great deep,” and these waters part for Him and His “redeemed” so that they “pass over” (Isa. 51:10). It is called a “Red Sea”-- and I do not mean a “Sea of Reeds”-- because it is RED. The water has been turned to blood—at least in God’s sight, for the BODY of His Son fills this vast universe (see Eph. 1:23), and the eternal blood that contained eternal life was shed to go up through this body of water. It is prefigured by “the Red Sea.” It must part for Christ to return, for He brings troops with Him when He comes (Hab. 3:10–16), and it was parted before when those troops ASCENDED at the Rapture. This is what Exodus 14 prefigured, among other things. Pharoah cannot get through the door (Heb. 10:19--20; John 10:1--9): he sinks (Exod. 15:19) and comes to earth knowing he has “but a short time” (Rev. 12:12). All the commentators miss all the Scriptural references, all of the doctrine, and all of the prophetic elements. They should have dropped out of the “ministry” years ago. (“If you don’t at first succeed—so much for sky diving.”) The Red Sea crossing is given in Isaiah 49:10 as a prelude to the Second Advent in the very next verse (vs. 11). The Red Sea crossing is given in Isaiah 63:12 in the context of the Second Advent (vss. 1--6), and the Jewish remnant in that passage (vss. 15--19) are Tribulation Jews awaiting deliverance by the Messiah (vss. 7--9). There is nothing, I say, “nothing” (absolutely NOTHING) on this earth that will prevent you from learning the Bible like sitting under a Bible corrector whose faith has been deposited in a lost pile of papers that disappeared from this earth more than 1,500 years ago. The “original” of Psalm 77 disappeared long before it could be put into any book of “original autographs” containing Isaiah, Malachi, Matthew, and Paul.\cite{Ruckman1992Psalms}}
